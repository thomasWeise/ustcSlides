\documentclass[mathserif]{beamer}%
%
%
\RequirePackage{./styles/slides}%
%
%
\title[%
short title%
]{%
long title%
}%
%
\author[Thomas Weise]{%
Thomas Weise\\%
\footnotesize{%
\mbox{\href{mailto:tweise@ustc.edu.cn}{tweise@ustc.edu.cn} $\cdot$} %
\mbox{\href{mailto:tweise@gmx.de}{tweise@gmx.de} $\cdot$} %
\mbox{\href{http://www.it-weise.de}{http://www.it-weise.de}}}}%
%
\institute[UBRI]{%
USTC-Birmingham Joint Res. Inst. in Intelligent Computation and Its Applications (UBRI)\\%
University of Science and Technology of China (USTC), Hefei 230027, Anhui, China%
}%
%
\date{\today}%
%
%
\begin{document}%
%
\startPresentation{}%
%
%
\AtBeginSection[]{%
\begin{frame}%
\frametitle{Section Outline}%
\tableofcontents[currentsection]%
\end{frame}%
}%
%
%
\section{Introduction}%
%
\begin{frame}%
\frametitle{\LaTeX}%
\begin{itemize}%
\item This is a theme for creating nice slides with \LaTeX\ and the beamer document class%
\end{itemize}%
\end{frame}%
%
\begin{frame}%
\frametitle{Structure}%
\begin{itemize}%
\item A presentation is structured into%
\begin{enumerate}%
\item a head, where you write information such as the title and your affiliation,%
\item a body, with the actual contents, and%
\item a foot, with the references and the good-bye slide%
\end{enumerate}%
\item The body can and should be divided into sections with the \texttt{{\textbackslash}section{\dots}} command%
\item Each section can contain multiple \texttt{frame}s, where each frame is one slide (which may be composed of several steps, see later)%
\end{itemize}%
\end{frame}%
%
\section{\LaTeX\ Commands}%
%
\begin{frame}%
\frametitle{Itemizations}%
This is an introduction to itemizations.%
\begin{itemize}%
\item We can have item lists with the \texttt{itemize} environment.%
\item Each item will then begin with an \texttt{{\textbackslash}item} command.%
\item Itemizations can also be nested:%
\begin{itemize}%
\item Like this.%
\item Which is nice too.
\end{itemize}%
\end{itemize}%
\end{frame}%
%
\begin{frame}%
\frametitle{Enumerations}%
This is an introduction to enumerations.%
\begin{enumerate}%
\item We can have numbered lists with the \texttt{enumerate} environment.%
\item Each item will then begin with an \texttt{{\textbackslash}item} command.%
\item Enumerations can also be nested:%
\begin{enumerate}%
\item Like this.%
\item Which is nice too.
\end{enumerate}%
\item We can also nest itemizations in them, and vice versa:%
\begin{itemize}%
\item See?%
\item And now another enumeration:%
\begin{enumerate}%
\item See?
\end{enumerate}%
\end{itemize}%
\end{enumerate}%
\end{frame}%
%
\begin{frame}%
\frametitle{Showing Stuff Step-Wise}%
Things can be shown step-wise\uncover<2->{:%
\begin{itemize}%
\item If you want to show a thing on the $n$\textsuperscript{th} \inQuotes{step} of a slide, you can wrap it into a command of the form \texttt{{\textbackslash}uncover<$n$->\{\dots\}}%
\item<3-> It then remains invisible until the $n$\textsuperscript{th} step is reached, but occupies space.%
\item<4-> If you want something to remain invisible and not occupy space, you can use \texttt{{\textbackslash}only<$n$->\{\dots\}}%
\item<5-> Actually, the stuff inside the \texttt{<\dots>} marks a range of steps.\only<6-9>{ Write%
\begin{itemize}%
\item \texttt{<10>} to show something \emph{only} at the 10\textsuperscript{th} step\uncover<7->{,}%
\item<7-> \texttt{<10->} to show something starting at the 10\textsuperscript{th} step and thereafter\uncover<8->{,}%
\item<8-> \texttt{<9-11>} to show something from the 9\textsuperscript{th} to the 11\textsuperscript{th} step\uncover<9->{,}%
\item<9-> \texttt{<4,9-11,14->} to show something from at the 4\textsuperscript{th}, the 9\textsuperscript{th} to the 11\textsuperscript{th}, and from the 14\textsuperscript{th} step onwards, and so on.%
\end{itemize}}%
\item<10-> Many commands, such as \texttt{{\textbackslash}item}, can take an argument of the form \texttt{<...>}, which makes their effect similar to be wrapped into an \texttt{{\textbackslash}uncover<$n$->\{\dots\}}.%
\item<11-> If you use \texttt{{\textbackslash}only} in your slides, you better use \texttt{{\textbackslash}begin\{frame\}[t]} instead of \texttt{{\textbackslash}begin\{frame\}} to begin a frame, or stuff will jump around like on this slide here.%
%
\end{itemize}%
}%
\end{frame}%
%
\begin{frame}%
\frametitle{Colors}%
\begin{itemize}%
\item You can use colors to color something%
\item<2-> You would use the \texttt{{\textbackslash}textcolor\{color\}\{text\}} command for this and, e.g., write \textcolor{red}{\texttt{{\textbackslash}textcolor\{red\}\{hello\}}}%
\item<3-> This command also takes an \texttt{<\dots>} argument, which allows you to do \textcolor<4,6>{green}{\texttt{{\textbackslash}textcolor<4,6>\{green\}\{hello\}}}%
\item<7-> You can also use the command \texttt{{\textbackslash}alert\{\dots\}} to mark something especially important.%
\item<8-> It, too, can take the \texttt{<\dots>} argument.%
\end{itemize}%
\end{frame}%
%
\begin{frame}%
\frametitle{References}%
Commands for references to literature (stored as Bib\TeX\ records in the file \texttt{bibliography/bibliography.bib}):%
\begin{itemize}%
\item\small numerical citations are done with \texttt{{\textbackslash}citep\{reference\}} and look like \inQuotes{blabla\citep{WGOEB}}%
\item\small numerical citations with non-breakable space in front are done with \texttt{{\textbackslash}scitep\{reference\}} and look like \inQuotes{blabla\scitep{WWCTL2016GVLSTIOPSOEAP}}, which is usually the better way%
\item\small citations with author names and numerical id are done as \texttt{{\textbackslash}citet\{reference\}} and look like \inQuotes{blabla \citet{WNSRG2008ATMFMOERANFL}}%
\item\small citations with author names and numerical id at the beginning of a sentence are done as \texttt{{\textbackslash}Citet\{reference\}} and look like \inQuotes{blabla. \Citet{WPG2009SRWVRPWEA}}%
\item\small citations with author names only are done as \texttt{{\textbackslash}citeauthor\{reference\}} and look like \inQuotes{blabla \citeauthor{WCTLTCMY2014BOAAOSFFTTSP}}%
\item\small citations with author names only at the beginning of a sentence are done as \texttt{{\textbackslash}Citeauthor\{reference\}} and look like \inQuotes{blabla. \Citeauthor{WCT2012EOPABT}}%
\end{itemize}%
\end{frame}%
%
\endPresentation%
%
%\appendices%
%
%
\end{document}%%
\endinput%
%